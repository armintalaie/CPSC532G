\begin{table*}
  \centering
  \begin{tabular}{lrrr}
    \hline
    \textbf{Variation} & \textbf{Total Samples} & \textbf{Correct Predictions} & \textbf{Success Rate} \\
    \hline
    Decreased Contrast & 108 & 73 & 0.676 \\
    Increased Contrast & 108 & 72 & 0.667 \\
    Vertically Stretched & 108 & 71 & 0.657 \\
    Horizontally Stretched & 108 & 70 & 0.648 \\
    Squeezed Vertically & 108 & 69 & 0.639 \\
    Decreased Brightness & 108 & 69 & 0.639 \\
    Decreased Sharpness & 108 & 69 & 0.639 \\
    Squeezed Horizontally & 108 & 67 & 0.620 \\
    Padded & 108 & 63 & 0.583 \\
    Decreased Saturation & 108 & 62 & 0.574 \\
    Increased Sharpness & 108 & 62 & 0.574 \\
    Selective Blur & 108 & 61 & 0.565 \\
    Baseline & 108 & 61 & 0.565 \\
    Increased Saturation & 108 & 61 & 0.565 \\
    Dominant Colors & 108 & 60 & 0.556 \\
    Adaptive Threshold & 108 & 59 & 0.546 \\
    Black and White & 108 & 59 & 0.546 \\
    Quantized Colors & 108 & 53 & 0.491 \\
    Increased Brightness & 108 & 50 & 0.463 \\
    Highlighted Intersections & 108 & 47 & 0.435 \\
    \hline
  \end{tabular}
  \caption{\label{tab:variation-results}
    Performance comparison across different image variations. All variations were tested on the same number of samples (108), with success rates ranging from 43.5\% to 67.6\%.
  }
\end{table*}
